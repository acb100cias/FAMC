\documentclass[]{article}
\usepackage[utf8x]{inputenc}
\usepackage[spanish]{babel}
\usepackage{amsmath,amsthm,amssymb}
\usepackage[]{geometry}

%%%%Titulo%%%%%%%%%%%%%%%%%%%%%%
%---------------------------------------------------------------------------------------
\title{Problemas de física atómica y materia condensada}
\author{Augusto Cabrera Manuel.}
\date{\today\\ \ser 3 \sem 2018-2 }

%-----------------------------------------------------------------------------------------
\newtheorem{definicion}{Definición}
\newtheorem{ejemplo}{Ejemplo}
\newtheorem{teorema}{Teorema}
\newtheorem{lema}{Lema}
\newtheorem{prop}{Proposición}
\newtheorem{af}{Afirmación}
\newtheorem{coro}{Corolario}
\newtheorem{obs}{Observación}
\newtheorem{casos}{Caso}
\newcommand\sem{\bf Semestre: }
\newcommand\gru{\bf Grupo: }
\newcommand\hor{\bf Horario: }
\newcommand\sal{\bf Salón: }
\newcommand\ser{\bf Serie: }

\renewcommand{\spanishrefname}{Bibliografía.}
\renewcommand{\spanishproofname}{Prueba.}
\def\sectionautorefname{section}



%%%%%%%%%%%%%%%%%%%%%%%

\begin{document}
\maketitle
\begin{enumerate}
\item Demuestre que todas las componentes  del momento angular orbital para una particula satisfacen
  \begin{equation*}
    \left[\hat{L}_{\alpha},f\left(\vec{r}\right)\right]=\hat{L}_{\alpha}f\left(\vec{r}\right)
  \end{equation*}
donde $\alpha$ toma valores en el conjunto $\{x,y,z\}$ y $f\left(\vec{r}\right)$ es una función de coordenadas derivable.
\item Demuestre que el operador  de momento angular orbital $vec{L}=\sum_{j=1}^{N} \vec{L}_j $ conmuta con el hamiltoniano sin interacción de N partículas
  \begin{equation*}
    \hat{H}=\sum^{N}_{j=1}\left(\frac{\hat{p}^2_j}{2m}-\frac{kZe^2}{r_j}\right)
  \end{equation*}
\item Demuestre que el operador de momento angular orbital $\vec{L}_{jk}=\vec{L}_j+\vec{L}_k$ conmuta con el operador de interacción electrón-electrón de Nelectrones
  \begin{equation*}
    \hat{V}=\frac{1}{2}\sum^2_{j\neq k}\frac{ke^2}{\left|\vec{r}_j-\vec{r}_k\right|}
  \end{equation*}
\item Utilizando los resultados de los problemas (2) y (3) demuestre que el operador  de momento angual orbital total para un átomo de N electrones conmuta con el hamiltoniano electrostático para dichos sistema dado por
  \begin{equation*}
    \hat{H}^{\left(0\right)}=\sum^{N}_{j=1}\left(\frac{\hat{p}^2_j}{2m}-\frac{kZe^2}{r_j}\right)+\frac{1}{2}\sum^2_{j\neq k}\frac{ke^2}{\left|\vec{r}_j-\vec{r}_k\right|}
  \end{equation*}
\item Demuestre  que para un átomo de N electrones el  operador de momento angualr total $\vec{J}=\vec{L}+\vec{S}$, donde $\vec{L}=\sum_{j=1}^N\vec{L}_j$ y $\vec{S}=\sum_{j=1}^N\vec{S}_j$, conmuta con el hamiltoniano de interacción espín-órbita dado por
  \begin{equation*}
    \hat{H}_{LS}=\sum_{j=1}^{N}F(r_j)\vec{L}_j\cdot\vec{S}_j
  \end{equation*}
\item  Demuestre que $\vec{J}$ conmuta con el hamiltoniano de un átomo de N electrones que incluye la interacción espín-órbita.
\item Determine los símbolos de términos que que represntan a las siguientes configuraciones electrónicas:
  \begin{enumerate}
  \item $np^1$
  \item $np^5$
  \item $nsnp$
  \item Determine en cada caso el termino que corresponde al estado base.
  \end{enumerate}
\item Demostrar que los símbolos de términos de las configuraciones  electrónicas $np^2$ y $np^4$
 coinciden.
\item Los símbolos de términos para una configuración electrónica $nd^2$ son: $^1S$, $^1D$, $^1G$, $^3P$ , $^3F$. Calcule los valores del momento angular total $J$ asociado con cada uno de éstos términos y determine cuál de ellos representa al estado base.
\item Demuestre que para funciones de estado hidrogenoides de un electrón el operador directo de energía potencial (de Hartree-Fock) es un operador central, esto es, actúa sólo  sobre la coordenada $r$ (distancia)
\end{enumerate}
\end{document}
