\documentclass[]{article}
\usepackage[utf8x]{inputenc}
\usepackage[spanish, es-tabla]{babel}
\usepackage{amsmath, amssymb, amsthm, amsfonts, latexsym}
\usepackage{extpfeil}
\usepackage[ left=2.5cm, right=2.5cm , bottom= 2.5cm, top=3cm]{geometry}
\usepackage{fancyhdr}
%    \pagestyle{fancy}
    
    %\fancyhead[R]{\thetitle\ \thepage} % Instead of this
    %d[R]{\theshorttitle } % do something like this
    
\usepackage{graphicx}
\usepackage{float}
\usepackage{xcolor}
\usepackage{pifont}
\usepackage{makeidx}
\makeindex
\usepackage{tocbibind}
\usepackage{wallpaper}
\usepackage{appendix}
\decimalpoint
\usepackage{pdfpages}



%---------------------------------------------------------------------------------------
\title{Física atómica y materia condensada}
\author{Augusto Cabrera Manuel.\\ Cubículo 317 Depto de Física FAcultad de Ciencias UNAM}
\date{\today\\ \sem 2018-2 \gru--- \hor--- \sal---}

%-----------------------------------------------------------------------------------------
\newtheorem{definicion}{Definición}
\newtheorem{ejemplo}{Ejemplo}
\newtheorem{teorema}{Teorema}
\newtheorem{lema}{Lema}
\newtheorem{prop}{Proposición}
\newtheorem{af}{Afirmación}
\newtheorem{coro}{Corolario}
\newtheorem{obs}{Observación}
\newtheorem{casos}{Caso}
\newcommand\sem{\bf Semestre: }
\newcommand\gru{\bf Grupo: }
\newcommand\hor{\bf Horario: }
\newcommand\sal{\bf Salón: }

\renewcommand{\spanishrefname}{Bibliografía.}
\renewcommand{\spanishproofname}{Prueba.}
\def\sectionautorefname{section}
%\allowdisplaybreaks
%%%%%%%%%%%%%%%%%%%%%%%%%%%%%%%%%%%%%%%%%%%%%%%%%%%%%%%%%%%%%%%%%%%%%%%%%
\begin{document}
\maketitle
\part*{Temario}

\section{Introducción}
\subsection{Materia condensada-Problema de muchos cuerpos}
\subsection{Aproximación de Born-Oppenheimer}
\subsection{Partículas idénticas y simetría de intercambio}
\subsection{Aproximación de Hartree-Fock}

\section{Átomos con $N$ electrónes}
\subsection{Hamiltoniano}
\subsection{Aproximación de potencial central de una cuasipartícula}
\subsection{Suma de momentos angulares}
\subsection{Estados cuánticos: COnfiguración, términos y símbolos de términos}
\subsection{Átomo en campos externos y reglas de selección}
\subsection{Espectros}

\section{Moléculas}
\subsection{Estados electrónicos de $H_2^{+}$: Aproximación MO-LCAO}
\subsection{Estados electrónicos de $H_2^{+}$: Aproximación Heitler-London}
\subsection{Enlace}
\subsection{Estados electrónicos de moléculas diatómicas con N electrones (configuración, términos y símbolos de términos)}
\subsection{Enlaces direccionales: Aproximación de orbitales híbridos}
\newpage
\section{Estado Sólido}
\subsection{Introducción}
\subsection{Enlaces: iónico, covalente, metálico}
\subsection{Estructura cristalina}
\subsection{Red de Bravais}
\subsection{Aproximación de Hartree-FockRed recíproca}
%\newpage
\section{Estados electrónicos}
\subsection{Modelo de cuasipartícula independiente}
\subsubsection{Gas de electrones libres}
\subsubsection{Gas de electrones dentro de un potencial periodico}
\subsection{Bandas electrónicas}
\subsection{Densidad de estados electrónicos}
\subsection{Energía total y calor específico electrónico}

\section{Dinámica de la red}
\subsection{Vibración y fonones en una red bidimensional diatómica}
\subsection{Densidad de estados vibracionales}
\subsection{Energía total y calor específico de la red}

\section{Sólidos en campos externos}
\subsection{Interacción del sólido con campo $\bar{E}$}
\subsection{Masa efectiva}
\subsection{Apantallamiento electrónico}
\subsection{Oscilaciones de plasma}
\subsection{Interacción de sólidos con campo $\bar{B}$}
\subsection{Interacción de sólidos con campos electromagnéticos $\bar{A}(\bar{r},t)$ ; $\Phi(\bar{r},t)$ }
\subsection{Excitaciones elementales y colectivas}

\newpage

\begin{thebibliography}{}
\bibitem{Levine} Ira N. Levine, Quantum Chemistry. Pearson-Prentice Hall, 2000.
\bibitem{Pilar} Frank L. Pilar, Elementary Quantum Chemistry, McGraw-Hill Book Company, 1968.
\bibitem{McQuarrie} Donald A. McCuarrie, Quantum Chemistry, University Science Books, 1983.
\bibitem{Lowe} Lowe, John P. Quantum Chemistry, Academic Press, 1978.
\bibitem{arkins} Atkins, P.W. Molecular Quantum Mechanics, Oxford University Press, 1983.
\bibitem{ashcroft} Ashcroft, Neil W. and N. David Mermin, Solid State Physics, Sounders College Publishing, 1976.
\bibitem{McKelvey} J. P. McKelvey, Física del estado sólido y semiconductores, ed Limusa, 1976.
\bibitem{Hall} H.E. HallFísica del estado sólido, ed. Limusa, 1978.
\bibitem{kittel} Charles Kittel, Introducción al estado sólido, Ed. Reverté, 1998.
\bibitem{ibach} Harald Ibach and Hans Luth, Solid State Physics, Springer Verlag, 1990.
\bibitem{kaxiras} Efthimios Kaxiras V. Atomic and electronic structure of solids, Cambridge University Press, 2003.
\bibitem{Madelung} Otfried Madelung, Introduction to Solid-State Theory, Springer-Verlag, 1978.
  \bibitem{McQuarrie2} Donald A. McCuarrie, Statistical Mechanics, HArper Collins Publishers, 1976.
\end{thebibliography}

\part*{Evaluación}
\begin{enumerate}
\item Tareas 50 $50\%$
\item Examenes $50\%$  (Al menos 3). Formado por preguntas y problemas.
\end{enumerate}

\end{document}
