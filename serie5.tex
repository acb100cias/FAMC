\documentclass[]{article}
\usepackage[utf8x]{inputenc}
\usepackage[spanish, es-tabla]{babel}
\usepackage{amsmath, amssymb, amsthm, amsfonts, latexsym}
\usepackage{extpfeil}
\usepackage[ left=2.5cm, right=2.5cm , bottom= 2.5cm, top=3cm]{geometry}
\usepackage{fancyhdr}
%    \pagestyle{fancy}
    
    %\fancyhead[R]{\thetitle\ \thepage} % Instead of this
    %d[R]{\theshorttitle } % do something like this
    
\usepackage{graphicx}
\usepackage{float}
\usepackage{xcolor}
\usepackage{pifont}
\usepackage{makeidx}
\makeindex
\usepackage{tocbibind}
\usepackage{wallpaper}
\usepackage{appendix}
\decimalpoint
\usepackage{pdfpages}



%---------------------------------------------------------------------------------------
\title{Problemas de física atómica y materia condensada}
\author{Augusto Cabrera Manuel.}
\date{\today\\ \ser 5 \sem 2018-2 }

%-----------------------------------------------------------------------------------------
\newtheorem{definicion}{Definición}
\newtheorem{ejemplo}{Ejemplo}
\newtheorem{teorema}{Teorema}
\newtheorem{lema}{Lema}
\newtheorem{prop}{Proposición}
\newtheorem{af}{Afirmación}
\newtheorem{coro}{Corolario}
\newtheorem{obs}{Observación}
\newtheorem{casos}{Caso}
\newcommand\sem{\bf Semestre: }
\newcommand\gru{\bf Grupo: }
\newcommand\hor{\bf Horario: }
\newcommand\sal{\bf Salón: }
\newcommand\ser{\bf Serie: }

\renewcommand{\spanishrefname}{Bibliografía.}
\renewcommand{\spanishproofname}{Prueba.}
\def\sectionautorefname{section}
%\allowdisplaybreaks
%%%%%%%%%%%%%%%%%%%%%%%%%%%%%%%%%%%%%%%%%%%%%%%%%%%%%%%%%%%%%%%%%%%%%%%%%
\begin{document}
\maketitle
\begin{enumerate}
\item En la molécula ion hidrógeno; supongamos que los núcleos se encuentran en las posiciones $\vec{R}_A=-\hat{e}_z\left(\frac{R}{2}\right)$ y $\vec{R}_B=\hat{e}_z\left(\frac{R}{2}\right)$ el electrón en $\vec{r}=\hat{e}_xx+\hat{e}_yy+\hat{e}_zz$ que $\vec{a}$ es el vector de proyección de $vec{r}$ en el plano $XY$ y tiene magnitud $a$. Utilizando las definiciones de las coordenadas elípticas cofocales $\mu$,$\nu$, $\phi$, calcule
  \begin{enumerate}
  \item Relaciones para $a$,$x$,$y$ y $z$ en términos de las coordenadas elípticas confocales
  \item Los factores de escala de transformación: $h_{\mu}$,$h_{\nu}$ y $h_{\phi}$ entre los dos sistemas de coordenadas.
  \item Una expresión para el volúmen en térmoinos de coordenadas elípticas confocales.
  \item El operador laplaciano en coordenadas elípticas confocales.
  \item La ecuación de Schrödinger estacionaria en coordenadas elípticas confocales.
  \item Demuestre que dicha ecuación es separable en este cojunto de coordenadas.
  \item Demuestre que la solución para la función que depende del ángulo $\phi$, es similar a la del electrón en el átomo de hidrógeno.
  \end{enumerate}
\item En la molécula de ión hidrógeno sobre el electrón  actúan fuerzas electrostáticas debido a los dos núcleos (protones). Como consecuencia, se produce una torca en la posición $\vec{r}$ del electrón. Así se produce un cambio en el tiempo sobre el momento angular orbital, esto es
$$
\frac{\mathrm{d}\vec{L}}{\mathrm{d}t}=\vec{\tau}
$$
\begin{enumerate}
\item Calcule la torca total que actúa sobre el electrón debido a los protones.
\item De acuerdo a las expresiones obtenidas ¿Qué puede decir sobre la evolución temporal de las componentes de $\vec{L}$?
\end{enumerate}
\item Para la molécula ion de hidrógeno demuestre que el conmutador $\left[\hat{H},\hat{L}_z\right]=0$ y por lo tanto $\hat{L}_z$ es constante de movimiento.
\item En la molécula ion de hidrógeno, demuestre que la densidad de probabilidad de encontrar al electrón en $z=0$ es:
  \begin{enumerate}
  \item Distinta de cero para el estado cuántico ``de enlace''.
  \item Cero para el estado cuántico ``de antienlace''.
  \end{enumerate}
\item  En la molécula de $H_2^+$ existe una operación de simetría llamada inversión respecto a un punto de inversión ( en este caso, el punto medio  de la distancia entre los dos átomos) a la operación la denotaremos por $\hat{I}_N$ y su efecto sobre el vector de posición del electrón $\vec{r}=(x,y,z)$ es el siguiente
$$
\hat{I}_N\vec{r}=-\vec{r}
$$
Demuestre que para  esta molécula se satisfacen las siguientes relaciones:
\begin{enumerate}
\item $Para H_2^+$, $I_N\phi_+(\vec{r})=\phi_+(\vec{r})$ y $I_N\phi_-(\vec{r})=-\phi_-(\vec{r})$ donde $\phi_+=\eta_+=\eta_g$ y$\phi_-=\eta_-=\eta_u$ son los estados moleculares de enlace y antienlace respectivamente para la molécula.
\end{enumerate}
\item Calcule las integrales siguientes que aparecen en el análisis de la molécula $H_2^+$. Comente sus resultados.
  \begin{enumerate}
  \item $S=\left<\xi_{1s}\left(\vec{r}_A\right)||\hat{I}||\xi_{1s}\left(\vec{r}_B\right)\right>$ donde $\hat{I}$ es el operador idéntico.
  \item $J=\left<\xi_{1s}\left(\vec{r}_A\right)||\left(\frac{e^2}{r_B}\right)||\xi_{1s}\left(\vec{r}_B\right)\right>$
  \item $K=\left<\xi_{1s}\left(\vec{r}_A\right)||\left(\frac{e^2}{r_A}\right)||\xi_{1s}\left(\vec{r}_B\right)\right>$
  \end{enumerate}
\item Muestre gráficamente que la energía del estado ``de enlace'' tiene un mínimo y encuentre un valor para el mismo.

\end{enumerate}
\end{document}
