\documentclass[]{article}
\usepackage[utf8x]{inputenc}
\usepackage[spanish]{babel}
\usepackage{amsmath,amsthm,amssymb}
\usepackage[]{geometry}

%%%%Titulo%%%%%%%%%%%%%%%%%%%%%%
%---------------------------------------------------------------------------------------
\title{Problemas de física atómica y materia condensada}
\author{Augusto Cabrera Manuel.}
\date{\today\\ \ser 1 \sem 2018-2 }

%-----------------------------------------------------------------------------------------
\newtheorem{definicion}{Definición}
\newtheorem{ejemplo}{Ejemplo}
\newtheorem{teorema}{Teorema}
\newtheorem{lema}{Lema}
\newtheorem{prop}{Proposición}
\newtheorem{af}{Afirmación}
\newtheorem{coro}{Corolario}
\newtheorem{obs}{Observación}
\newtheorem{casos}{Caso}
\newcommand\sem{\bf Semestre: }
\newcommand\gru{\bf Grupo: }
\newcommand\hor{\bf Horario: }
\newcommand\sal{\bf Salón: }
\newcommand\ser{\bf Serie: }

\renewcommand{\spanishrefname}{Bibliografía.}
\renewcommand{\spanishproofname}{Prueba.}
\def\sectionautorefname{section}

%%%%%%%%%%%%%%%%%%%%%%%

\begin{document}
\maketitle
\begin{enumerate}
\item El estado dinámico de un sistema cuántico esta descrito por la función $\Psi(\vec{r},t)$ y su hamiltoniano es $\hat{H}$. Sea $\hat{G}_1$ un operador asociado con una cantidad física del sistema y que actúa sobre la función de estado. Demuestre que el cambio en el tiempo del valor esperado de $\hat{G}_1$ en el estado $\Psi(\vec{r},t)$ es
  \begin{equation*}
    \frac{\mathrm{d}\left<\hat{G}_1\right>}{\mathrm{d}t}=\left(\frac{i}{\hbar}\right)\left<\left[\hat{H},\hat{G}_1\right]\right>+\left<\frac{\partial\hat{G}_1}{\partial t}\right>
  \end{equation*}
con
\begin{equation*}
  \left<\hat{G}_1\right>=\int\Psi^*(\vec{r},t)\hat{G}_1\Psi(\vec{r},t)\mathrm{d}\tau
\end{equation*}
donde $\mathrm{d}\tau$ representa el elemento de volúmen del sistema en el espacio de configuración y $\left[\hat{H},\hat{G}_1\right]$ es el conmutador de los dos operadores.
\item Considere un conjunto de N partículas idénticas, descrito por un hamiltoniano que sólo actúa sobre coordenadas espaciales de la forma
  \begin{equation*}
    \hat{H}(\vec{r}_1,\vec{r}_2\dots ,\vec{r}_j\dots ,\vec{r}_k\dots ,\vec{r}_N)=\sum_{j=1}^N\left(\frac{\hat{p}_j^2}{2m}+V\left(\vec{r}_j\right)\right)+\frac{1}{2}\sum_{j\neq k}V\left(\left|\vec{r}_j-\vec{r}_k\right|\right)
  \end{equation*}
Sea $\hat{P}_{jk}$ el operador que intercambia las coordenadas espaciales y de espín de las partículas $j$ y $k$, esto es
\begin{equation*}
  \hat{P}_{jk}\varphi\left(x_1, x_2\dots ,x_j\dots ,x_k\dots ,x_N\right)=\varphi\left(x_1, x_2\dots ,x_j\dots ,x_k\dots ,x_N\right)
\end{equation*}
con $x_j=(\vec{r}_j,S_j)$

Demostrar que:
\begin{enumerate}
\item $\hat{P}_{jk}$ es hermitiano.
\item $\hat{P}_{jk}$ conmuta con el hamiltoniano.
\item $\dfrac{\mathrm{d}\hat{P}_{jk}}{\mathrm{d}t}=0$.
\item La simetría de la función de estado $\varphi\left(x_1, x_2\dots ,x_j\dots ,x_k\dots ,x_N\right)$ es una constante de movimiento.
\end{enumerate}
\item Construya una función antisimétrica para un sistema de 3 electrones sin interacción entre ellos.
\item Para un átomo de N electrones, la función de onda tipo Determinante de Slater puede escribirse como
  \begin{equation*}
    \Phi(x_1,x_2\dots ,x_N)=\hat{A}\left[\xi_1\left(x_1\right)\xi_2\left(x_2\right)\dots \xi_N\left(x_N\right)\right]=\hat{A}\prod_{j=1}^N\xi_j(x_j)
  \end{equation*}
donde el operador $\hat{A}$ esta definido por
\begin{equation*}
  \hat{A}=\left(\frac{1}{N!}\right)^{\frac{1}{2}}\sum_{R=0}^{N!-1}(-1)^R\hat{P}_R
\end{equation*}
donde $hat{P}_R$ representa la permutación $R$-ésima del producto de espín-orbitales de una partícula que intervienen en el desarrollo del Determinante de Slater.

Demostrar que el operador  $\hat{A}$ tiene las siguientes propiedades:
\begin{enumerate}
\item $\hat{A}^2=\sqrt{N!}\hat{A}$
\item $\hat{A}=\hat{A}^t$, es autoadjunto.
\item $\hat{A}\hat{\Omega}=\hat{\omega}\hat{A}$ donde $\hat{\Omega}$ es un operador simétrico ante el intercambio de coordenadas espín-espaciales de 2 partículas.
\end{enumerate}
\item Revisar y reproducir la deducción de las funciones de distribución Térmico-estadisticas de Fermi-Dirac y bose-Einstein de partículas sin interacción entre ellas.
\end{enumerate}
\end{document}
