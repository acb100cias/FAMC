\documentclass[]{article}
\usepackage[utf8x]{inputenc}
\usepackage[spanish, es-tabla]{babel}
\usepackage{amsmath, amssymb, amsthm, amsfonts, latexsym}
\usepackage{extpfeil}
\usepackage[ left=2.5cm, right=2.5cm , bottom= 2.5cm, top=3cm]{geometry}
\usepackage{fancyhdr}
%    \pagestyle{fancy}
    
    %\fancyhead[R]{\thetitle\ \thepage} % Instead of this
    %d[R]{\theshorttitle } % do something like this
    
\usepackage{graphicx}
\usepackage{float}
\usepackage{xcolor}
\usepackage{pifont}
\usepackage{makeidx}
\makeindex
\usepackage{tocbibind}
\usepackage{wallpaper}
\usepackage{appendix}
\decimalpoint
\usepackage{pdfpages}



%---------------------------------------------------------------------------------------
\title{Problemas de física atómica y materia condensada}
\author{Augusto Cabrera Manuel.}
\date{\today\\ \ser 7 \sem 2018-2 }

%-----------------------------------------------------------------------------------------
\newtheorem{definicion}{Definición}
\newtheorem{ejemplo}{Ejemplo}
\newtheorem{teorema}{Teorema}
\newtheorem{lema}{Lema}
\newtheorem{prop}{Proposición}
\newtheorem{af}{Afirmación}
\newtheorem{coro}{Corolario}
\newtheorem{obs}{Observación}
\newtheorem{casos}{Caso}
\newcommand\sem{\bf Semestre: }
\newcommand\gru{\bf Grupo: }
\newcommand\hor{\bf Horario: }
\newcommand\sal{\bf Salón: }
\newcommand\ser{\bf Serie: }

\renewcommand{\spanishrefname}{Bibliografía.}
\renewcommand{\spanishproofname}{Prueba.}
\def\sectionautorefname{section}
%\allowdisplaybreaks
%%%%%%%%%%%%%%%%%%%%%%%%%%%%%%%%%%%%%%%%%%%%%%%%%%%%%%%%%%%%%%%%%%%%%%%%%
\begin{document}
\maketitle

\begin{enumerate}
\item \begin{enumerate}
\item Demuestre que el movimiento de rotación y vibración de dos núcleos en una molécula diatómica, que se encuentran a una distancia $R$, entre ellos, en general variable; puede transformarse en el movimiento relativo de rotación y vibravión de los núcleos, entre sí, y el moviemiento de traslación del centro de masa del sistema.
\item Escriba en coordenadas esféricas la ecuación de Schrödinger para el movimiento rotaciona y vibracional de la masa reducida  que representa a dos núcleos de la molécula diatómica, en un potencial que depende sólo de la coordenada relativa.
\item Proponga una solución que separe las coordenadas  y escriba las ecuaciones resultantes.
\item Encuentre ls valores propios de la energía fe un rotor no-rígido, que consiste de dos núcñleos con masas $M_1$ y $M_2$ unidos a través de un potencial central tipo oscilador armónivo y libre de girar en el espacio. 
\end{enumerate}
\item Para el potencial de Morse definido coo
$$
V(R)= D_e\left(1-e^{-\alpha\left(R-R_e\right)}\right)^2
$$
$D_e$ es la profundidad del pozo, respecto a la energía de dos átomos de hidrógeno libres. $R_e$ es la distancia  internuclear de equilibrio, $\alpha$ es un parámetro.
\begin{enumerate}
\item Encontrar una expresión para la constante de fuerza, en la aproximacióm de oscilador armónico, definida por
$$
k=\left(\frac{\mathrm{d}^2V(R)}{\mathrm{d} R^2}\right)_{R=R_e}
$$
\item Para el hidrógeno molecular se tiene
\end{enumerate}
\end{enumerate}

\end{document}