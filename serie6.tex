\documentclass[]{article}
\usepackage[utf8x]{inputenc}
\usepackage[spanish, es-tabla]{babel}
\usepackage{amsmath, amssymb, amsthm, amsfonts, latexsym}
\usepackage{extpfeil}
\usepackage[ left=2.5cm, right=2.5cm , bottom= 2.5cm, top=3cm]{geometry}
\usepackage{fancyhdr}
%    \pagestyle{fancy}
    
    %\fancyhead[R]{\thetitle\ \thepage} % Instead of this
    %d[R]{\theshorttitle } % do something like this
    
\usepackage{graphicx}
\usepackage{float}
\usepackage{xcolor}
\usepackage{pifont}
\usepackage{makeidx}
\makeindex
\usepackage{tocbibind}
\usepackage{wallpaper}
\usepackage{appendix}
\decimalpoint
\usepackage{pdfpages}



%---------------------------------------------------------------------------------------
\title{Problemas de física atómica y materia condensada}
\author{Augusto Cabrera Manuel.}
\date{\today\\ \ser 6 \sem 2018-2 }

%-----------------------------------------------------------------------------------------
\newtheorem{definicion}{Definición}
\newtheorem{ejemplo}{Ejemplo}
\newtheorem{teorema}{Teorema}
\newtheorem{lema}{Lema}
\newtheorem{prop}{Proposición}
\newtheorem{af}{Afirmación}
\newtheorem{coro}{Corolario}
\newtheorem{obs}{Observación}
\newtheorem{casos}{Caso}
\newcommand\sem{\bf Semestre: }
\newcommand\gru{\bf Grupo: }
\newcommand\hor{\bf Horario: }
\newcommand\sal{\bf Salón: }
\newcommand\ser{\bf Serie: }

\renewcommand{\spanishrefname}{Bibliografía.}
\renewcommand{\spanishproofname}{Prueba.}
\def\sectionautorefname{section}
%\allowdisplaybreaks
%%%%%%%%%%%%%%%%%%%%%%%%%%%%%%%%%%%%%%%%%%%%%%%%%%%%%%%%%%%%%%%%%%%%%%%%%
\begin{document}
\maketitle
\begin{enumerate}
\item Como consecuencia del método variacional en la molécula de hidrógeno se encuentran dos estados moleculares representados por $\phi_{_{+}}(1,2)$ en el que la molécula es estable  y otro representado por $\phi_{_{-}}(1,2)$ en el que la molécula es inestable. Demuestre que la función de espín $\chi_{_{+}}$ asociada con el estado molecular $\phi_{_{+}}(1,2)$ es la función propia del operador $S_z$, con valor propio $M_s=0$ mientras que $\chi_{_{-}}$ asociada con el estado molecular $\phi_{_{-}}(1,2)$ es la función propia del mismo operador, con valores propios $M_s=1,0.-1$.
\item Calcule las siguientes integrales que aparecen en el cálculo de la energía y los estados moleculares  mediante el método variacional en la aproximación de Heitler y London o de enlace covalente en la molecula de Hidrógeno:
  \begin{enumerate}
  \item  $J_1=\left<\xi_A(r_1)\xi_B(r_2)\left|\dfrac{ke^2}{r{B1}}\right|\xi_A(r_1)\xi_B(r_2)\right>$
  \item $J_2=\left<\xi_A(r_1)\xi_B(r_2)\left|\dfrac{ke^2}{r{A2}}\right|\xi_A(r_1)\xi_B(r_2)\right>$
  \item $J_3=\left<\xi_A(r_1)\xi_B(r_2)\left|\dfrac{ke^2}{r{B1}}\right|\xi_A(r_2)\xi_B(r_1)\right>$
  \item $J_4=\left<\xi_A(r_1)\xi_B(r_2)\left|\dfrac{ke^2}{r{A2}}\right|\xi_A(r_2)\xi_B(r_1)\right>$
  \end{enumerate}
puede utilizar resultados de la molécula ion hidrógeno siempre que lo haga con cuidado.
\item Puede extenderse el método variacional tomando como función de prueba una combinacion lineal de funciones de estado atómicas para $N$ electrones en el potencial de $q$ núcleos, esto es 
$$
\phi(1,2,3,\dots ,N)=\sum_{j=1}^qC_j\Omega_j(1,2,3,\dots N)
$$
donde $\phi(1,2,3,\dots ,N)$ representa el estado molecular y $\Omega_j(1,2,3,\dots N)$ representa los estados atómicos de $N$ electrones respectivamente y son ortonormales. Los coeficientes $C_j$ son parámetros por determinar. A las funciones $phi$ se les exige que satisfagan condiciones de ortonormalización.
\begin{enumerate}
\item Construya la funcional $F[C_J]$ que debe usarse en el método variacional.
\item Demostrar que la condición $\delta F=0$ conduce a un sistema homogéneo de $q$ ecuaciones algebráicas en los parámetros y la energía.
\item Mencione esquematica y culaitativamente el camino a seguir para resolver  el sistema homogéneo de ecuaciones  y como determinar los valores de energía del estado molecular  y las funciones de onda correspondientes
\end{enumerate}
\item En la molécula de $H_2$ existe una operación de simetría llamada inversión resécto al punto de inversión (en este caso el punto medio de la distancia entre dos átomos) a esta operacoión la denotaremos por $\hat{I}_N$ y su efecto sobre un vector de posición de cada uno de los electrones $\vec{r}=(x,y,z)$ es el siguiente
$$
\hat{I}_N\vec{r}=-\vec{r}
$$
Demuestre que para la molécula referida se satisfacen las siguientes relaciones:
\begin{eqnarray*}
I\phi_{_{+}}(1,2)&=&\phi_{_{+}}(1,2)\\
I\phi_{_{-}}(1,2)&=&-\phi_{_{-}}(1,2)
\end{eqnarray*}
donde $\phi_{_{+}}$ y $\phi_{_{-}}$ son los estados moleculares de enlace y de antienlace respectivamente para la molécula.
\item Una molécula diatómica homonuclear en su estado base molecular tiene la siguiente configuración electrónica
$$
1\sigma^2_g1\sigma^2_u2\sigma^2_g2\sigma^2_u1\pi^4_u3\sigma^2_g1\pi^2_g
$$
\begin{enumerate}
\item ¿Cuál es el orden del enlace?
\item ¿Cuál es la multiplicidad del espín?
\item ¿Cuál es la diferencia sobre la estabilidad de la molécula al ionizarla eliminando un electron del orbital molecular $1\pi_g$ o al hacerlo del orbital molecular $3\sigma_g$?
\end{enumerate}
\item
  \begin{enumerate}
  \item Investigar en qué consiste el método de construcción de orbitales híbridos y aplicarlo para:
  \item Construir orbitales híbridos tipo $sp_z$,$\xi_1$ y $\xi_2$.
  \item Construir orbitales híbridos tipo $sp^2$,$\xi_1$, $\xi_2$ y $\xi_3$ usando los orbitales $s$, $p_x$ y $p_z$
  \end{enumerate}

\end{enumerate}
\end{document}
