\documentclass[]{article}
\usepackage[utf8x]{inputenc}
\usepackage[spanish]{babel}
\usepackage{amsmath,amsthm,amssymb}
\usepackage[]{geometry}

%%%%Titulo%%%%%%%%%%%%%%%%%%%%%%
%---------------------------------------------------------------------------------------
\title{Problemas de física atómica y materia condensada}
\author{Augusto Cabrera Manuel.}
\date{\today\\ \ser 2 \sem 2018-2 }

%-----------------------------------------------------------------------------------------
\newtheorem{definicion}{Definición}
\newtheorem{ejemplo}{Ejemplo}
\newtheorem{teorema}{Teorema}
\newtheorem{lema}{Lema}
\newtheorem{prop}{Proposición}
\newtheorem{af}{Afirmación}
\newtheorem{coro}{Corolario}
\newtheorem{obs}{Observación}
\newtheorem{casos}{Caso}
\newcommand\sem{\bf Semestre: }
\newcommand\gru{\bf Grupo: }
\newcommand\hor{\bf Horario: }
\newcommand\sal{\bf Salón: }
\newcommand\ser{\bf Serie: }

\renewcommand{\spanishrefname}{Bibliografía.}
\renewcommand{\spanishproofname}{Prueba.}
\def\sectionautorefname{section}



%%%%%%%%%%%%%%%%%%%%%%%

\begin{document}
\maketitle
\begin{enumerate}
\item Demuestre que el determinante de Slater  para el estado base  del helio puede representarse como el producto de dos funciones : una parte dependiente de coordenadas espaciales que es simétrica y la otra que depende de coordenadas de espín que es antisimétrica, respecto al intercambio de coordenadas espín-espaciales de los dos electrones
\item Cuál es el valor del espín total del átomo de Helio
  \begin{enumerate}
  \item En estado base
  \item En un estado excitado
  \end{enumerate}
\item Determine la energía del estado base del helio mediante:
  \begin{enumerate}
  \item El método de perturbaciones, realizando los desarrollos con detalle.
  \item El método variacional, usando la carga del núcleo apantallada como parámetro. Realice las integrales necesarias con el mayor detalle posible.
  \end{enumerate}
\item Para el átomo de Litio:
  \begin{enumerate}
  \item Escriba  la función de onda del estado base
  \item Demuestre que dicha función es función propia del operador proyección $z$ del operador del espín total $S_z=S_{1z}+S_{2z}+S_{3z}$
  \item Encuentre los valores propios del operador $\hat{S}_z$ en el estado base.
  \end{enumerate}
\item Para un sistema de $3$ electrones ; calcule el valor propio  del operador $S_z$ en el estado representado por la función de espín
  \begin{equation*}
    \Xi=\frac{1}{3!}\left[\alpha(1)\alpha(2)\beta(3)+\alpha(1)\beta(2)\alpha(3)+\beta(1)\alpha(2)\alpha(3)\right]
  \end{equation*}
donde $\alpha$ representa el estado con proyección de espín $\frac{\hbar}{2}$ y $\beta$ el estado de proyección de espín $-\frac{\hbar}{2}$.
\item Para un átomo de 3 electrones:
  \begin{enumerate}
  \item Si exigimos que el determinante de Slater esté normalizado ¿Qué condiciones deben exigirse a las funciones espín-orbitales?Considerando que los espín-orbitales y sus ocupaciones son
    \begin{equation*}
      \xi_1(x_1)=\xi_1(vec{r}_1)\alpha_1\qquad\xi_2(x_2)=\xi_2(vec{r}_2)\beta_2\qquad\xi_3(x_3)=\xi_3(vec{r}_3)\alpha_3
    \end{equation*}
Calcular simbólicamente los valores esperados  para cada uno de los estados con hamiltonianos de una partícula, haciendo ver explícitamente  el efecto de la parte de espín.
\item hacer un calculo similar en cada una de las integrales directas y de intercambio.

  \end{enumerate}
\item En este ejercicio $\alpha$ y $beta$ representan funciones propias del operador $\hat{S}_z$ mientras que $f$ y $g$ son funciones generales de coordenadas espaciales , de las cuales no se tiene más información. Las etiquetas  etre paréntesis indican coordenadas de partículas. De las siguientes funciones
  \begin{eqnarray*}
    f(1)g(2)\alpha(1)\alpha(2)\\
    f(1)f(2)\left[\alpha(1)\beta(2)-\beta(1)\alpha(2)\right]\\
    f(1)f(2)f(3)\beta(1)\beta(2)\beta(3)\\
    \left[f(1)g(2)-g(1)f(2)\right]\left[\alpha(1)\beta(2)-\beta(1)\alpha(2)\right]\\
    r^2_{12}e^{-\alpha\left[r_1+r_2\right]}\\
    e^{-\alpha\left[r_1-r_2\right]}
  \end{eqnarray*}
cuáles son:

    \begin{enumerate}
    \item Completamente simétricas;
    \item Completamente asimétricas;
    \item En cuáles falta informaciónpara clasificarlas en los términos anteriores
    \end{enumerate}
  \item
    \begin{enumerate}
    \item Calcule en forma explícita y directa el promedio cuántico del operador $\left(\dfrac{e^2}{\vec{r}_2-\vec{r}_3}\right)$ con las funciones spin-orbitales $\hat{A}\left[\xi_1(x_1)\xi_2(x_2)\xi_3(x_3)\right]$
    \item Calcule los términos del promedio cuántico calculado en (a), distintos de cero.
    \end{enumerate}

\end{enumerate}
\end{document}
