\documentclass[]{article}
\usepackage[utf8x]{inputenc}
\usepackage[spanish]{babel}
\usepackage{amsmath,amsthm,amssymb}
\usepackage[]{geometry}

%%%%Titulo%%%%%%%%%%%%%%%%%%%%%%
%---------------------------------------------------------------------------------------
\title{Problemas de física atómica y materia condensada}
\author{Augusto Cabrera Manuel.}
\date{\today\\ \ser 4 \sem 2018-2 }

%-----------------------------------------------------------------------------------------
\newtheorem{definicion}{Definición}
\newtheorem{ejemplo}{Ejemplo}
\newtheorem{teorema}{Teorema}
\newtheorem{lema}{Lema}
\newtheorem{prop}{Proposición}
\newtheorem{af}{Afirmación}
\newtheorem{coro}{Corolario}
\newtheorem{obs}{Observación}
\newtheorem{casos}{Caso}
\newcommand\sem{\bf Semestre: }
\newcommand\gru{\bf Grupo: }
\newcommand\hor{\bf Horario: }
\newcommand\sal{\bf Salón: }
\newcommand\ser{\bf Serie: }

\renewcommand{\spanishrefname}{Bibliografía.}
\renewcommand{\spanishproofname}{Prueba.}
\def\sectionautorefname{section}



%%%%%%%%%%%%%%%%%%%%%%%

\begin{document}
%\maketitle
\begin{enumerate}
\item Calcule la energía del primer estado excitado del átomo de helio, usando perturbaciones independientes del tiempo, para estados degenerados.
\item Una base  de vectores propios para $\hat{J}^2=\left(\hat{J}_1+\hat{J}_2\right)^2$ y su proyección $\hat{J}_z=\hat{J}_{z_1}\hat{J}_{z_2}$ puede representarse como $\left||jm\right>$, de modo que $\hat{J}^2\left||jm\right>=j\left(j+1\right)\hbar^2\left||jm\right>$ y $\hat{J}_z\left||jm\right>0m\hbar\left||jm\right>$. 

El valor propio máximo de la proyección en unidades de $\hbar$ es $m_{\max}=(m_1+m_2)_{\max}=j_1+j_2$
\begin{enumerate}
\item ¿Cual es el valor máximo de $j$?
\item Encontrar el valor mínimo de $j$ utilizando el hecho de que el número de estados en la representación de estados propios  de $\hat{J}_1$ y $\hat{J}_2$ representado por $\left||j_1m_1\right>\left||j_2m_2\right>$ y en la representación de $\left||jm\right>$ se conserva; esto es 
\begin{equation*}
\sum_{j_{\min}}^{j_{max}}\left(2j+1\right)=(2j_1+1)(2j_2+1)
\end{equation*}
\end{enumerate}
\item Dedúzca el símbolo de término para el estado base para los siguientes átomos
  \begin{enumerate}
  \item Zirconio, cuya configuración electrónica es  $[KR](4d)^2(5s)^2$.
  \item Paladio, cuya configuración electrónica es $[KR](4d)^{10}$
  \end{enumerate}
\item Sugiera justificadamente con qué átomos podría realizarse un experimento que usando efecto Zeeman normal permita determinar la razón $\frac{e}{m}$ para el electron
\item Para el experimento de Stern-Gerlach, átomos de Plata se calienta  a $1000 K$. Después de un haz delgado de dichos átomos  se pasan a través  de u campo magnético que varía en el espacio con un gradiente  de $0.75\frac{Weber}{cm}$ en una distancia de $10 cm$, en la dirección del eje $Z$. El haz después de salir  del campo viaja $12 cm$ antes de golpear una placa fotográfica.
  \begin{enumerate}
  \item Calcule la separación de las dos componentes del haz sobre la pantalla. Use la velocidad promedio del haz de la relación $\frac{1}{2}m<v>^2=\frac{3}{2}K_BT$.
  \item Si el experimento de realiza con los siguientes átomos: $Ca( ^1S_0)$, $Ti( ^3F_2)$, $As( ^4S_{\frac{3}{2}})$ ;¿Cuántas líneas  o manchas se observarán en la pantalla?
  \end{enumerate}
\item Calcule el factor  de Landé, $g(L,S,J)$ y el momento magnético $\vec{\mu}_J$ de átomos en los estados $^2D_{\frac{5}{2}}$, $^2D_{\frac{3}{2}}$ y $^2F_{\frac{7}{2}}$.
\item Demuestre que cuando un átomo con momento angular total $\vec{J}$, se coloca  en un campo magnético constante $\vec{B}=B_0\hat{e}_z$, el momento angular precesa alrededor del campo magnético con una frecuencia $\omega=\frac{eB_0}{2m}g(L,S,J)$.
\item En los siguientes $3$ casos , el átomo tiene una transición entre dos estados cuánticos y se encuentra sometido a un campo magnético constante. Encontrar en cada caso, el número de líneas  que se observan en precencia de dicho campo.
  \begin{enumerate}
  \item Cadmio en la transición $^1P-^1D$ cuya longitud de onda  sin campo es $\lambda=6438.47$ angtroms.
  \item Doblete de sodio con longitudes de onda $\lambda_1=5889.96$ angtroms y $\lambda_2=5895.93$ angtroms, correspondientes a transiciones  $^2P_{\frac{3}{2}}-^2S_{\frac{1}{2}}$ y $^2P_{\frac{1}{2}}-^2S_{\frac{1}{2}}$
  \item La línea con $\lambda=4722.16$ angtroms del zinc correspondiente a la transición $^3P_{1}-^3S_{1}$.
  \end{enumerate}

\end{enumerate}
\end{document}
